\documentclass{article}

\def\w{\omega}
\def\wp{\omega^\prime}
\def\wc{{\omega_{\rm C}}}
\def\ket{\rangle}
\def\bra{\langle}
\def\H{\hat{H}}
\def\P{\hat{P}_{\rm occ}}
\def\E{\varepsilon}
\def\vp{{v^\prime}}
\def\q{{\bf q}}
\def\s{\sigma}
\def\k{{\bf k}}
\def\qp{{\bf q^\prime}}
\def\G{{\bf G}}
\def\Gp{{\bf G^\prime}}
\def\rt{\tilde{r}}
\def\pt{\tilde{p}}
\def\rb{{\bf r}}
\def\rp{{\bf r^\prime}}
\def\rpp{{\bf r^{\prime\prime}}}
\def\rppp{{\bf r^{\prime\prime\prime}}}
\def\mo{$\overline{1}$}
\def\mt{$\overline{2}$}
\def\S{\mathcal{S}}
\def\Gs{\mathcal{G}}
\def\v{\mathbf{v}}
\def\symm{\left\{\mathcal{S}|\mathbf{v}\right\}}

\begin{document}
\section{Expressions}
This is just a brief note summarizing what I have
calculated so far for $\mu$ and $\mu^{*}$.
We denote a symmetry operation of the Crystal as::
%
\begin{equation}
  \label{eq:symmoperation}
  \symm\rb = \S\rb + \v.
\end{equation}

We write the matrix elements with the screened Coulomb interaction as:
%
\begin{eqnarray}
\label{eq:coulmat}
W_{n\k n'\k'} = \int \psi^{*}_{n'\k'}(\rb)\psi_{n\k}(\rb)W(\rb,\rp) \nonumber\\
\times\psi^{*}_{n'-\k'}(\rp)\psi_{n-\k}(\rp) \rm{d}\rb \rm{d}\rp.
\end{eqnarray}


The isotropic Coulomb pseudopotential $\mu$ is given:
%
\begin{eqnarray}
\label{eq:musymm}
\mu =\frac{1}{N(0)}\frac{1}{\Omega}\sum_{\q\in{\rm IBZ}}\sum_{S}
\frac{{\rm w}_{\q}}{N_{\rm group}}\sum_{\k\in {\rm IBZ}}\nonumber{\rm w}_{\k}\sum_{n n' \G \G'}\\ 
f^{*}_{n\k n'\k'}(\G)\frac{4\pi e^{2}}{|\q+\G|^{2}}
\epsilon_{\G\G'}^{-1}(\q)f_{n\k n'\k'}(\G')
\end{eqnarray}
%

In Eq.~\ref{eq:musymm}, $\Omega$ is the volume of the crystal unit cell, 
$\rm{w}_{\k}$ $\rm{w}_{\q}$, are the weights (per spin) of the $\q$ and $\k$ 
points in the Irreducible Brillouin Zone (IBZ). In addition $\k'=\k-\S^{-1}\q$.
We have defined $f_{n'\k',n\k}$ as:
%
\begin{equation}
f_{n'\k',n\k}(\G) = \int e^{-i\G\cdot\rb} \psi_{n'\k'}(\rb)\psi_{n\k}(\rb) \rm{d}\rb
\end{equation}

%To calculate the anisotropic tensor I have simply dropped the sum over $n$ and $n'$
%and accumulated Eq.~\ref{eq:musymm} in a matrix $\mu_{n n'}$:
%
%\begin{eqnarray}
%\label{eq:munnp}
%\mu_{n n'} = \frac{1}{N(0)}\frac{1}{\Omega}\sum_{\q\in{\rm IBZ}}\sum_{S}
%\frac{{\rm w}_{\q}}{N_{\rm group}}\sum_{\k\in {\rm IBZ}}\nonumber{\rm w}_{\k}\sum_{\G \G'}\\ 
%f^{*}_{n\k n'\k'}(\G)\frac{4\pi e^{2}}{|\q+\G|^{2}}
%\epsilon_{\G\G'}^{-1}(\q)f_{n\k n'\k'}(\G')
%\end{eqnarray}

\section{Pb}
In Table~\ref{tab:pb} I calculated the isotropic
Coulomb pseudopotential parameters using the Lindhard
and the full LDA dielectric function calculated using
the Sternheimer equation. It appears the local field
effects are more significant in Pb enhancing the screening 
by around $10\%$. For Pb the $N(0)$ I have calculated is
$0.242$ states per eV per atom per spin.

\begin{table}
\begin{center}
\begin{tabular}{l c c c c}
             & $\mu$  & $E_{F}$  & $\omega_{c}$  & $\mu^{*}$ \\
\hline
Lindhard     & 0.239  & 11.74    & 0.035         & 0.100 \\
LDA          & 0.191  & 11.74    & 0.035         & 0.091 \\ 
\end{tabular}
\caption{Coulomb pseudopotential for Pb calculated using
the Lindhard dielectric function and the LDA screening.\label{tab:pb}}
\end{center}
\end{table}

%In addition I have calculated the anisotropic $\mu_{nn'}$ matrix
%in Table~\ref{tab:pbaniso}. It appears there are three bands
%participating in the Coulomb pseudopotential around the Fermi
%energy.
%
%\begin{table}
%\begin{center}
%\begin{tabular}{c c c}
%Band 2     & Band 3    & Band 4    \\
%0.0207440  & 0.0331581 & 0.0024041 \\
%0.0297328  & 0.0949253 & 0.0061428 \\
%0.0022238  & 0.0060766 & 0.0005877 \\
%\end{tabular}
%\caption{Anisotropic parameters for Pb. \label{tab:pbaniso}}
%\end{center}
%\end{table}
%

\section{MgB$_{2}$}
For MgB$_{2}$ I have prepared Table \ref{tab:mgb2} for a range of different
isotropic $\mu^{*}$, electronic energy scales, and
vibrational energy scales. In each case I've calculated $\mu^*$ using
the LDA electronic bandwidth, the ideal electron gas, and the
lowest plasmon frequency of MgB$_2$ and for a range of vibrational
cutoffs $\omega_{c}$. For MgB$_2$ the N(0) I have calculated is
$0.35$ states per eV per spin.

I think the plasmon energy is the most appropriate
choice for the electronic energy scale in this system
and might be interesting to discuss in the context of the 
Eliashberg equations and the superconducting temperature.
%
\begin{table}
\begin{center}
\begin{tabular}{l c c c c}
\hline
\hline
             & $\mu$  & $E_{F}$  & $\omega_{c}$  & $\mu^{*}$ \\
LDA          & 0.224  & 12.43    & 0.5    & 0.130 \\ 
Electron Gas & 0.224  & 9.76     & 0.5    & 0.134 \\
Plasmon      & 0.224  & 2.50     & 0.5    & 0.165 \\
\hline
LDA          & 0.224  & 12.43    & 0.35   & 0.124 \\
Electron Gas & 0.224  & 9.76     & 0.35   & 0.128 \\
Plasmon      & 0.224  & 2.50     & 0.35   & 0.156 \\
\hline
LDA          & 0.224  & 12.43    & 0.086  & 0.106 \\
Electron Gas & 0.224  & 9.76     & 0.086  & 0.109 \\ 
Plasmon      & 0.224  & 2.50     & 0.086  & 0.128 \\
\end{tabular}
\caption{$\mu^{*}$ calculated using different electronic and vibrational
energy scales. \label{tab:mgb2}}
\end{center}
\end{table}

In Table~\ref{tab:mgb2aniso} I give the anisotropic parameters for $\mu$. For
MgB$_{2}$ there are 8 electrons in the unit cell. Bands 1 and 2
are completely filled and Bands 3, 4, and 5 are only partially
filled and contribute to the Fermi surface. 

By placing a restriction on the integration in $\k$ space we are
able to separate $\mu$ into the contributions from $\pi$ and $\sigma$
bands. 

\begin{table}
\begin{center}
\begin{tabular}{l c c}
$\sigma$  & 0.0719  & 0.02593   \\
$\pi$     & 0.0239  & 0.09823   \\
\end{tabular}
\caption{Anisotropic $\mu$ for MgB$_2$ without the band dependent density of states. \label{tab:mgb2aniso}}
\end{center}
\end{table}

%
If we follow the convention of Moon et. al. they define
the anisotropic $\mu$:
%
\begin{equation}
\mu_{nn'} = \frac{1}{N(0)}\sum_{\k\k'}\frac{\delta(\epsilon_{n\k}-\epsilon_{F})}{N_{n}} W_{n\k n'\k'}
\delta(\epsilon_{n'\k'}-\epsilon_{F})
\end{equation}
%
where $N_{n}$ is the density of states at the fermi level for band $n$.
Including a band density of states $N_{\sigma} = 2.01$, and $N_{\pi}=2.75$ 
states/Ry per atom per spin we find Table \ref{tab:bdos}:
%
\begin{table}
\begin{center}
\begin{tabular}{l c c}
$\sigma$  & 0.036  & 0.014 \\
$\pi$     & 0.009  & 0.038 \\
\end{tabular}
\caption{Anisotropic $\mu$ for MgB$_2$ including $N_{n}$, the band dependent density of states
. \label{tab:bdos}}
\end{center}
\end{table}
%

Comparing the ratios in Table~\ref{tab:bdos} we find 
$\mu_{\sigma\sigma}/N_{\sigma}:\mu_{\pi}/N_{\pi}$ is $1.28:1.00$ suggesting weaker
dielectric screening for the $\sigma$ orbitals. This can be directly compared
with the ratio Moon et. al. find $1.31:1.00$. We can also compare
$\mu_{\sigma\sigma}:\mu_{\pi\pi}:\mu_{\sigma\pi}:\mu_{\pi\sigma}$,
$0.036:0.038:0.014:0.009$ Reducing this to common fractions
we find $4.0:4.22:1.56:1$. This too can be directly compared with
Moon et. al. relative numers of $3.54:3.69:1.38:1.00$. So clearly the present calculations
suggest a slightly higher level of anisotropy then previously found but 
of similar relative magnitudes.

I am also still not sure what the preferred procedure is for renormalizing
the anisotropic $\mu$ parameter. Tomorrow I will look at the Mazin
paper where they given an expression for the anisotropic $\mu^{*}$.

\section{Renormalized $\mu^{*}$}
Mazin gives the extension to a multiband case:%
%
\begin{equation}
\mu^{*}_{ij} = \mu_{ij} - \sum_{n}\mu_{in} \ln(W_{n}/\omega_{c})\mu^{*}_{nj}
\end{equation}
%
\section{Conclusion}
%
Those are the results for Mg$B_{2}$ and Pb which I have acquired so far.
Hopefully $CaC_{6}$ will be completed soon as well. Numerically I am happy
with the level of convergence in each case. I think the small discrepancy
with the previous MgB$_{2}$ calculations might be related to convergence with
the sum over states. My thinking is the sum over
states might understimate $\epsilon$ and result in a corresponding underestimate of the
screening. This is consistent with the fact that using the Lindhard dielectric
function I calculate the Classical result for $\mu^{*}$, Al and Pb of 0.100. When using the
full dielectric matrix I get slightly smaller values for Al $0.096$ and Pb $0.091$. 

\end{document}
