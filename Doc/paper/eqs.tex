\documentclass{article}
%
% warning: if you redefine \r you will have troubles with the angstrom,
% which is internally defined as \r{A}
%

\def\w{\omega}
\def\>{\rangle}
\def\<{\langle}
\def\H{\hat{H}}
\def\E{\varepsilon}
\def\vp{{v^\prime}}
\def\q{{\bf q}}
\def\G{{\bf G}}
\def\Gp{{\bf G^\prime}}
\def\rt{\tilde{r}}
\def\pt{\tilde{p}}
\def\r{{\bf r}}
\def\rp{{\bf r^\prime}}
\def\rpp{{\bf r^{\prime\prime}}}
\def\rppp{{\bf r^{\prime\prime\prime}}}

\begin{document}

From Eq. 13.19b anmd 14.8 of HL, after splitting the sum in occupied and empty
(we remain with two sums), and swapping m,n indices in the second sum:
  \begin{eqnarray}
  W(\r,\rp;\w) & = & v(\r,\rp) \nonumber \\
     & + & \int d\rppp 2\sum_{n{\rm occ}} \psi_n^\star(\rppp) 
       \Big( \sum_{m{\rm un}} \frac{\int d\rpp \psi_m^\star W_{\r,\w} \psi_n}{\E_n-\E_m-\w}\psi_m(\rppp)\Big) v(\rppp,\rp) \nonumber \\
     & + & \int d\rppp 2\sum_{n{\rm occ}} \psi_n(\rppp) 
       \Big( \sum_{m{\rm un}} \frac{\int d\rpp \psi_n^\star W_{\r,\w} \psi_m}{\E_n-\E_m+\w}\psi_m^\star(\rppp)\Big) v(\rppp,\rp) \nonumber \\
  \end{eqnarray}
By using perturbation theory (nondegenerate) the previous equation becomes
  \begin{eqnarray}
  W(\r,\rp;\w) & = & v(\r,\rp) \nonumber \\
    & + & \int d\rpp v(\rpp,\rp) \, 2\sum_{n{\rm occ}} \Big[\psi_n^\star(\rpp) \Delta \psi_{n,+\w}(\rpp) + 
      \psi_n(\rpp) \Delta \psi_{n,-\w}^\star(\rpp) \Big]  \nonumber 
  \end{eqnarray}
where $\Delta \psi_{n,\pm\w}$ are the solutions of the systems ($v$ stands for ``valence'')
  \begin{equation}
  (H-\E_v+\w\phantom{i}) \Delta \psi_v^+(\w) = -(1-P_{\rm occ}) ( W_{\r,\w} \psi_v )
  \end{equation}
  \begin{equation}
  (H-\E_v-\w^\star) \Delta \psi_v^-(\w) = -(1-P_{\rm occ}) ( W^\star_{\r,\w} \psi_v ).
  \end{equation}
We can also define an induced charge density (this is not a proper charge but is analogous
to the charge induced at zero frequency):
  \begin{equation}
  \Delta n (\r;\w) = 2\sum_v \Big[\psi_v^\star(\r) \Delta \psi_v^+(\r;\w) +
      \psi_v(\r) \Delta \psi_v^{-\star}(\r;\w) \Big]
  \end{equation}
In this case we can rewrite the first equation as
  \begin{equation}
  W(\r,\rp;\w) = v(\r,\rp) + \int d\rpp \Delta n (\rpp;\w) \, v(\rpp,\rp)
  \end{equation}

\end{document}
